\begin{itemize}
    \item $sg$ 值為 $0$ 代表先手必敗
    \item 當前 $sg$ 值 $=$ 可能的後繼狀態的 $mex$ (例如拿一個或拿兩個, 就等於兩者的 $sg$ 值 $mex$), 若有互相依賴就兩個後繼狀態 $xor$ 當作一組 $sg$ 值 (例如切開成兩半, 只算一次)
    \item 單組基礎 $nim$ 的 $sg$ 值為本身的原因: $f(0) = 0, f(1) = mex(f(0)) = 1, f(2) = mex(f(0), f(1)) = 2 ...$, 都是自己
    \item 多組賽局可以把 $sg$ 值 $xor$ 起來, 當成最後的 $sg$ 值, nim 也是一樣, 且由於 $xor$ 性質, 如果可以快速知道 $sg(1) \^ sg(2) \^ ... \^ sg(n)$, 就可以用 $xor$ 性質處理不連續組合
\end{itemize}